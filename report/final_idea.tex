\documentclass[11pt,a4paper]{article}

\usepackage[margin=1in]{geometry}
\usepackage{setspace}
\usepackage{hyperref}
\usepackage{enumitem}

\setstretch{1.15}

\title{\textbf{Context-Aware FinCommerce Engine using Vector Memory with Qdrant}}
\author{}
\date{}

\begin{document}
\maketitle

\section*{Project Overview}

This project presents a \textbf{Context-Aware FinCommerce Engine} that enables intelligent product discovery and personalized recommendations by combining \textbf{semantic understanding} with \textbf{user-specific financial context}.

Instead of relying solely on keyword-based search or static recommendation rules, the platform uses \textbf{vector embeddings stored in Qdrant} to capture semantic meaning across products, user profiles, financial constraints, and interaction history.

At query time, the system retrieves semantically relevant products and then \textbf{re-ranks them using affordability constraints and personal preferences}, ensuring that recommendations are not only relevant but also financially realistic for each user.

The solution demonstrates how \textbf{Qdrant can act as a vector-native memory layer} for real-time, context-aware decision-making in FinCommerce applications.

\section*{Problem Statement}

Modern e-commerce and fintech platforms struggle to deliver recommendations that are both \textbf{semantically relevant} and \textbf{financially appropriate} for individual users.

Traditional recommendation systems typically:
\begin{itemize}
    \item Rely on keyword-based search or collaborative filtering
    \item Ignore real-time financial constraints such as account balance or credit limits
    \item Treat personalization, affordability, and user behavior as disconnected systems
\end{itemize}

This results in unrealistic product suggestions, poor user experience, and reduced conversion rates.

There is a growing need for a \textbf{unified, vector-native system} capable of reasoning simultaneously over \textbf{products, user preferences, financial context, and interaction history}, while supporting fast retrieval and filtering at scale.

\section*{Use Case Solved}

The project addresses the \textbf{Context-Aware FinCommerce Recommendation} use case.

When a user submits a natural-language query (e.g., \emph{``Laptop for machine learning under 1500''}), the system must return products that:
\begin{itemize}
    \item Are semantically relevant to the user’s intent
    \item Match the user’s personal preferences (brands, categories)
    \item Respect financial constraints such as available balance or credit limits
    \item Adapt based on historical interactions
\end{itemize}

Two users issuing the same query may receive different recommendations depending on their financial situation and preferences. This enables \textbf{real-time, personalized, and financially realistic discovery}, which is not achievable with traditional search engines.

\section*{How Qdrant Is Used}

Qdrant serves as the \textbf{central vector memory and retrieval engine} of the system.

\subsection*{Vector Collections}

The system uses multiple Qdrant collections, each representing a different type of memory:

\begin{itemize}
    \item \textbf{Product Collection}: Stores semantic embeddings of product descriptions along with structured metadata such as price, category, brand, and availability.
    \item \textbf{User Profile Collection}: Represents user preferences and personalization signals.
    \item \textbf{Financial Context Collection}: Stores financial constraints such as balances, income, and credit limits.
    \item \textbf{Interaction Memory Collection}: Captures historical user interactions and queries.
\end{itemize}

\subsection*{Query-Time Workflow}

At query time, the system performs the following steps:
\begin{enumerate}
    \item The user query is embedded into a vector representation.
    \item Qdrant retrieves the most semantically similar products using vector similarity search.
    \item Payload filters enforce structured constraints such as price limits and availability.
    \item Results are re-ranked using affordability and preference-based scoring.
\end{enumerate}

\subsection*{Why Qdrant}

Qdrant enables:
\begin{itemize}
    \item High-dimensional semantic search over multimodal data
    \item Hybrid retrieval combining vector similarity and structured filtering
    \item Low-latency performance suitable for real-time personalization
    \item A scalable and explainable architecture for FinCommerce intelligence
\end{itemize}

Rather than acting as a passive database, Qdrant functions as an \textbf{active intelligence layer} powering context-aware recommendations.

\section*{Alignment with Hackathon Use Case}

This project directly addresses \textbf{Use Case 2: Context-Aware FinCommerce Engine}, showcasing how Qdrant can enable personalized, constraint-aware discovery and recommendation using vector-native memory at scale.

\end{document}
